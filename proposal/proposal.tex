\documentclass[12pt]{article}

\usepackage{amsmath}
\usepackage[margin = 1in]{geometry}
\usepackage{graphicx}
\usepackage{booktabs}
\usepackage{natbib}

\usepackage[colorlinks=true, citecolor=blue]{hyperref}

\title{Proposal: Studying How Baseball Players Perform After Signing A New Contract}
\author{Jason Boghosian\\
  Department of Statistics\\
  University of Connecticut
}

\begin{document}
\maketitle


\paragraph{Introduction}
My paper will be an extensive study on the phenomenon of baseball players performing significantly worse immediately after changing teams. I am currently reading the book How Markets Fail by John Cassidy, wherein he discusses how “hidden information” can affect any given market. I will apply this concept to professional baseball’s free agency market and show how teams suffer from insufficient information problems when shopping for new players. I have been enamored with baseball and its underlying analytics for many years and I feel as though this topic mixes well the theoretical and empirical approaches to the sport.

\paragraph{Specific Aims}
My research hypothesis will be as follows: Major league baseball players see a decline in performance after switching teams in free agency when compared with players that re-sign a new contract with their current team. This hypothesis tests a long-posited theory that teams who sign new players don’t always get the player they bargained for. The concept is extremely relevant because teams rely on free agency to build the nucleus of their teams. Per Spotrac, over $3 billion in new contracts were signed in the last baseball offseason, signifying the importance of calculated and sound free agency decisions by both players and teams.

\paragraph{Data}
Sampling scheme: I will include in my data only batters and starting pitchers who signed a free agent contract between the end of the 2021 MLB season and the beginning of the 2022 MLB season. This sample will consist of 58 batters and 37 starting pitchers, for a total of 95 players. The variables that will be included are whether or not the player changed teams and whether they are a batter or a pitcher. 

The dependent variables will be an assortment of baseball-related statistics. For batters, I will use plate appearances, at bats, hits, doubles, triples, home runs, runs batted in, strikeouts, stolen bases, stolen base attempts, wins over replacement, runs over replacement, games played, number of days on the injured list, batting average, slugging percentage, home run percentage, strikeout percentage, and stolen base percentage. 

For pitchers, I will use innings pitched, batters faced, wins, hits surrendered, home runs surrendered, walks, earned runs, strikeouts, hit by pitches, wins over replacement, runs over replacement, games played, days on the injured list, earned run average, walks and hits per innings pitched, opponent batting average, strikeouts per 9 innings, hits surrendered per 9 innings, home runs surrendered per 9 innings, walks per 9 innings, and strikeouts per walk.

\paragraph{Research Design and Methods}
I will begin by separating the players into two main groups. The first group will be the players that switched teams, while the second group will be the players that stayed with their current team. For all of the aforementioned metrics that can be categorized as counting statistics, I will run a one-way analysis of variance between the 2021 and 2022 seasons for two groups. For the metrics that are proportions or ratios, I will run a difference of proportions hypothesis test with a Z-distribution. A P-value less than my alpha value, which will likely be 0.05, will denote a rejection of the null hypothesis and acceptance of the conclusion that the group’s 2022 season was significantly worse than the group’s 2021 season.

\paragraph{Discussion}
I expect to find that players who switched teams have a significant dropoff in more statistical categories than players who remained on their current teams. This is because when teams let players go to another team, there is often an underlying motive that the buying team is unaware of- troubling injury history, worrying underlying metrics, or other hidden reasons. Reaching this conclusion would corroborate a longstanding belief around the baseball world that players typically fail to perform after signing new contracts with new teams. Failure to reach this conclusion would also be a useful outcome- it would be equally important to learn that the 2021-2022 batch of free agents did not experience any dropoff, a testament to modern technological and statistical capabilities. 

\paragraph{Conclusion}
My research will aim to either corroborate or disprove the theory that I have co-opted into my research hypothesis. By comparing year-over-year player data, I can definitively prove the existence of trends in player performance. While I will not be able to assert that there is a 100% linkage between the statistical findings and my predicted reasoning for the results, I can give some credence to my hypothesis and provide a look at sweeping league-wide trends in baseball’s most high-profile league.


\bibliography{course-paper-or-presentation/PROPOSAL/citations.bib}
\bibliographystyle{chicago}

\end{document}
