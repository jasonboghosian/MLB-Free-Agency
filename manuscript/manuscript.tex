\documentclass[10pt]{article}
\usepackage[explicit]{titlesec}
\setlength{\parindent}{0pt}
\setlength{\parskip}{1em}
\usepackage{hyphenat}
\usepackage{ragged2e}
\RaggedRight

\usepackage{garamondx}
\usepackage[T1]{fontenc}
\usepackage{amsmath, amsthm}
\usepackage{graphicx}
\usepackage{hyperref}
\usepackage{xcolor}
\newcommand{\jy}[1]{\textcolor{orange}{JY: (#1)}}
\usepackage{booktabs}

\usepackage{soul}
\setul{.6pt}{.4pt}

\usepackage{geometry}
\geometry{vmargin={1in,1in}, hmargin={.75in, .75in}}
\usepackage{fancyhdr}
\pagestyle{fancy}
\pagenumbering{gobble}
\renewcommand{\headrulewidth}{0.0pt}
\renewcommand{\footrulewidth}{0.0pt}

\usepackage[labelfont={footnotesize,bf} , textfont=footnotesize]{caption}
\captionsetup{labelformat=simple, labelsep=period}
\newcommand\num{\addtocounter{equation}{1}\tag{\theequation}}
\renewcommand{\theequation}{\arabic{equation}}
\makeatletter
\renewcommand\tagform@[1]{\maketag@@@ {\ignorespaces {\footnotesize{\textbf{Equation}}} #1.\unskip \@@italiccorr }}
\makeatother
\setlength{\intextsep}{10pt}
\setlength{\abovecaptionskip}{2pt}
\setlength{\belowcaptionskip}{-10pt}

\renewcommand{\textfraction}{0.10}
\renewcommand{\topfraction}{0.85}
\renewcommand{\bottomfraction}{0.85}
\renewcommand{\floatpagefraction}{0.90}

\titleformat{\section}
  {\normalfont}{\thesection}{1em}{\MakeUppercase{\textbf{#1}}}
\titlespacing\section{0pt}{0pt}{-10pt}
\titleformat{\subsection}
  {\normalfont}{\thesubsection}{1em}{\textit{#1}}
\titlespacing\subsection{0pt}{0pt}{-8pt}
\renewcommand{\baselinestretch}{1.15}

\makeatletter
\newcommand\sixteen{\@setfontsize\sixteen{17pt}{6}}
\renewcommand{\maketitle}{\bgroup\setlength{\parindent}{0pt}
\begin{flushleft}
\sixteen\bfseries \@title
\medskip
\end{flushleft}
\textit{\@author}
\egroup}
\makeatother

%\usepackage[biblabel]{cite}
\usepackage[sort&compress]{natbib}
\setlength\bibindent{2em}
\makeatletter
\renewcommand\@biblabel[1]{\textbf{#1.}\hfill}
\makeatother
\renewcommand{\citenumfont}[1]{\textbf{#1}}
\bibpunct{}{}{,~}{s}{,}{,}
\setlength{\bibsep}{0pt plus 0.3ex}

\usepackage{float}

\title{Free Agency in Baseball: MLB Teams Aren’t Getting The Player They Bargained For}
\author{Jason Boghosian\\
  Department of Statistics\\
  University of Connecticut
}

\begin{document}
\maketitle

\section*{Abstract}

Every winter, professional baseball teams bid for the services of the MLB’s most talented hitters. Baseball’s top stars often 
sign contracts for upwards of half a decade, guaranteeing them tens, if not hundreds of millions of dollars. However, some of 
these contracts work out better than others. When teams shell out inordinate amounts of money, they are gambling on their
investment panning out; when it doesn’t, the consequences can be drastic. The conventional wisdom in baseball states that
players who sign contracts with new teams tend to underperform expectations in their new city. Mostly supported by anecdotal 
and circumstantial evidence, this theory can be scientifically put to the test. By comparing player performance from 2021 to
2022 for players that switched baseball clubs this past winter, we can either support or contradict this long-repeated 
conjecture. In this paper, we will explore the differences from year to year in numerous key baseball statistics. For each
significant statistical finding, practical explanations will be offered based on existing literature and practical causal 
arguments.

\section*{Keywords}

Baseball Contracts; Baseball Statistics; Free Agency; Injuries; Player Decline; Player Durability; Player Performance

\section{Introduction}

On May 5th, 2022, Boston Red Sox second baseman Trevor Story walked back to the dugout after a strikeout, his fourth such 
result of the night. Boos rained down on him from Red Sox fans who had, to this point, watched their beloved team muster zero 
runs through seven innings. Story had made his Red Sox debut just 27 days ago after signing a six-year, 140 million dollar 
contract over the winter with Boston. How did the multimillion-dollar experiment go sideways so quickly?

Fans of major league baseball have heard a version of this story many times before. A player destined for stardom seeks a 
change of scenery and then finds themself struggling to meet expectations in their new city. Particularly in the 2022 MLB 
season, fans of several teams found themselves wildly disappointed with shiny new acquisitions that they were positive would 
generate elite output. However, these negative sentiments gave me pause. Sports fans are among the most emotionally fraught 
and reactive individuals that I can think of. Red Sox fans had written off the 29-year-old Story before he had even spent a 
full month playing with his new club. I figured that there had to be a statistical method to test whether these fans’ gripes 
had some merit or whether they were merely the reactionary groans of disillusioned and bitter sports fans. Consequently, it 
is that question that I seek to answer with the methods presented in this paper. 

Author John Cassidy briefly explored a version of this phenomenon in his book entitled How Markets Fail, an economics book with 
an added statistical and mathematical flair. In Chapter 12, he references a study done by Kenneth Lehn, a financial economist 
at the Washington University School of Business. He found that output dropped among players who switched teams, with the main 
culprit being health. Players who moved spent, on average, nearly eight more days on the disabled list than players who re-upped 
their contracts with their existing team. His theory was that ball clubs seeking to poach players from other teams suffered from 
a key handicap- they had vastly less information about said players than did the existing teams of the players. I’ll explore 
this angle later on, but Cassidy and Lehn provided a strong introductory look into the matter. 

Stephen M. Garcia et al. explored the many intricacies of baseball free agency in a 2020 economic journal. They found an 
increasing unwillingness for high-ranking teams to pursue high-profile free agents. Teams seem to have soured on the 
idea of the splashy, big-money winter acquisitions. Haunted by memories of major flops from years past, more teams are looking 
for production from their young and developing homegrown talents rather than outside, expensive, established talent. Another 
trend they found is that when other teams bid for the services of their players, teams are becoming less likely to provide a 
counteroffer that would retain that player. Teams becoming more willing to part with their players aligns with the hypothesis 
that a player’s existing team knows more about them than any other team does. A player’s true state of fitness is technically 
hidden information- the consequences of nagging injuries often don’t present themselves until a player ages, long after a team 
has already taken a gamble on them.

Baseball teams shell out billions of dollars (3.2 billion total in 2022) in guaranteed money every winter in their quests to 
hoist the World Series trophy. It’s a high-stakes gambling game that produces more losers than winners, and can either save or 
cost people their jobs. The contribution of this paper is a more advanced look into what Cassidy and Lehn contributed. The rest 
of the paper is organized as follows. I will first lay out the source of my data and how it was cleaned and organized. I will 
then describe and justify the methods I used to compare player performance across the selected years. There will then be a 
report of all output tables, a walkthrough of all statistical computations, and finally in-depth analysis of my findings 
and what they may mean for the sport at large. 

\section{Data}

The data that has been collected for this study came mainly from Baseball Reference and Spotrac, two websites that contain 
nearly every morsel of baseball data one could desire. Baseball Reference mostly provided the true statistics and game-by-game 
breakdowns, while Spotrac mainly provided contract and free agency information. 

To begin, I had to decide who, or what, to sample. Because we are comparing 2021 statistics to 2022 statistics, I had to look 
at only players who switched teams in free agency between those two seasons. For the sake of consistency, we will refer to the 
period between the 2021 and 2022 seasons as the “2022 offseason”. 64 batters were free agents during the 2022 
offseason. Of those 64, precisely 50 signed major-league contracts with a team that was different from the team they were on 
at the end of the 2021 season. Spotrac lists six other players who technically switched teams, however these players did not 
switch teams until after the commencement of the 2022 season, making them ineligible for this study.

Once I had my sample of 50 players, I had to then choose which statistics I was going to use to measure player performance. 
The first group of statistics I analyzed we will refer to as “counting statistics”. These statistics are simple: they merely 
count the number of events of interest that occur in a player’s season. Runs (R) refers to the number of times a player touched 
home plate and scored a run for their team. Hits (H) refers to the number of times a player hit the ball into fair territory 
and reached base without a defensive error. Doubles (2B) refers to the number of hits that resulted in the batter reaching 
second base. Triples (3B) refers to the number of hits that resulted in the batter reaching third base. Home runs (HR) refers 
to the number of times a player hit a ball over the home run fence in fair territory, awarding them all four bases. Runs batted 
in (RBI) refers to the number of runs that scored as a direct result of the player being up to bat. Stolen bases (SB) refers 
to the number of times a player was on base and advanced by taking a base when they weren’t entitled to it. Finally, total 
bases (TB) refers to the number of bases gained by a batter as a direct result of their hitting.

The next kind of statistic is referred to as a “value-added statistic”. These statistics are technically counting statistics, 
however they measure a player’s added value above a hypothetical league-average player. Wins above replacement (WAR) measures 
a player's value by deciphering how many more wins they’re worth than a replacement-level player at the same position. Runs 
above replacement (RAR) measures a player’s value by deciphering how many more runs they’re worth than a replacement-level 
player at the same position.

The next two statistics are still technically counting statistics, but I portioned them off because they contribute to a 
different narrative than the rest of the counting statistics. They will be referred to as “health statistics”. Games (GM) 
refers to the number of games a player appeared in. Days on the injured list (IL) refers to the number of days a player spent 
on either the 10-day, 15-day, or 60-day injured list. When a player is on the injured list, they are ineligible to play due to 
injury.

The final group of statistics can be referred to as proportion statistics. These statistics are the only ones used in this 
study that can rise and fall, and are not dependent on the amount of game participation. For example, a player that plays 150 
games has a much better chance of having more hits than a player who plays 50 games. The following statistics do not follow 
the same rule. Batting average (AVG.) is the proportion of at bats that result in a hit for the player. Slugging percentage 
(SLG.) is calculated by dividing total bases by the number of at bats. Home run percentage (HR%) is the proportion of a 
player’s plate appearances that result in a home run. Walk percentage (BB%) is the proportion of a player’s plate appearances 
that result in a walk. Strikeout percentage (SO%) is the proportion of a player’s plate appearances that result in a strikeout. 
Finally, stolen base percentage (SB%) is the proportion of a player’s attempts to steal a base that are successful. 

All of the aforementioned statistics are extremely common in baseball data discourse. If the goal of the study is to measure 
player performance, these are the best parameters to use. When a player is at bat, the television broadcast of the game will 
display several key statistics, all of which are included in the study: batting average, home runs, and runs batted in. The 
chosen data will help answer the research question by providing a multifaceted look at all aspects of a player’s offensive 
game. Contact, power, speed, and discipline are all measured in at least one of the given variables.

\section{Methods}

The basic statistical mechanism used in this study was an analysis of variance (ANOVA). By utilizing a one-way ANOVA, I was able to 
compare means for all of the aforementioned variables from 2021 to 2022. The one-way ANOVA assumes that the data follows a 
normal distribution, which is an acceptable approximation given the large sample size of players. Each counting statistic, 
value-added statistic, and health statistic was run through this one-way analysis of variance test. For each of these 
statistics, the null and research hypotheses followed the same general form. The null hypothesis was that the mean value 
among all 50 players for these statistics in 2021 was equal to the mean value among the same 50 players in 2022. This 
hypothesis is the one we seek to reject, thus we need a research hypothesis that has the burden of proof. We must be able 
to measure the strength of the research hypothesis via our P-value. Therefore, our research (alternate) hypothesis is as 
follows. The mean value among all 50 players for these statistics in 2021 was not equal to the mean value among the same 50 
players in 2022. We are seeking to uncover whether there was a significant change in player performance between the two years, 
so these hypotheses will suffice. 

The statistical method changes slightly when analyzing the proportion statistics. Means are not very useful when comparing 
these kinds of statistics. For example, if Player 1 gets 10 hits in 20 tries (a .500 batting average) and Player 2 gets 0 hits 
in 5 tries (a .000 batting average), the mean of their batting averages will be .250. However, if you pool their attempts, 
which gives Player 1 additional weight due to the larger sample, they have a combined 10 hits in 25 tries, for a batting 
average of .400. This .400 figure is much more accurate than the .250 figure because it creates a weighted mean that gives 
each player a weight that is directly proportional to their number of attempts. The same tactic was applied to our proportion 
statistics. For these figures, I created a pooled number that used totals for the necessary variables rather than calculating 
a proportion statistic for each individual player. With the unweighted method, the mean batting average for our players in 2021 
was .237. This number is arrived at by adding together all 50 batting averages and dividing by 50. However, as we have 
established, this is a foolhardy strategy. Instead, I divided the total number of hits by our players in 2021 (4391) by the 
total number of times they were at bat in 2021 (17,591) to arrive at a more accurate weighted average of nearly .250. It would 
make sense that this number is higher than the original .237 because players that get hits more frequently are also more likely 
to have more at bats. 

Once I calculated these pooled proportions, I utilized a 2-sample Z test for the difference between proportions. The null 
hypothesis is only slightly changed from before. In this case, the null hypothesis is that there is no significant difference 
between the proportion from 2021 and the proportion from 2022. Our research hypothesis, again with the burden of proof, would 
then be that there is, in fact, a significant difference between the proportion from 2021 and the proportion from 2022. We can 
apply these hypotheses to all of the proportion statistics to seek out significant differences.

\section{Simulation and Application}

The following tables contain the results of the one-way analysis of variance for our counting statistics. The first column 
describes what the statistic is that is being tested. The second column contains the average number of that statistic that 
the 50 players recorded during the 2021 season. The third column contains the average number of that statistic that the 50 
players recorded during the 2022 season. The fourth column contains the raw increase or decrease in that statistic from 2021 
to 2022. The fifth column contains the percentage increase or decrease in that statistic from 2021 to 2022. The sixth column 
contains the P-value for the difference in the two means as calculated by SAS 9.4. The final column contains the significance 
marking for the statistic. A statistic with one asterisk indicates rejection of the null hypothesis at alpha = 0.10. A 
statistic with two asterisks indicates rejection of the null hypothesis at alpha = 0.05. A statistic with three asterisks 
indicates rejection of the null hypothesis at alpha = 0.01.

\begin{table}[h!]
  \begin{center}
    \caption{Counting Statistics}
    \label{tab:table1}
    \begin{tabular}
      \textbf{Stat} & \textbf{2021} & \textbf{2022} & \textbf{Diff.} & \textbf{PctgChg} & \textbf{P-Value} & \textbf{Signif.}
      \hline
      R & 51.34 & 36.94 & -14.4 & -28.05 & 0.0202 & ** \\
      H & 87.82 & 71.54 & -16.28 & -18.54 & 0.1038 &  \\
      2B & 16.64 & 13.76 & -2.88 & -17.31 & 0.1686 &  \\
      3B & 1.38 & 0.92 & -0.46 & -33.33 & 0.1206 &  \\
      HR & 14.84 & 9.82 & -5.02 & -33.83 & 0.0172 & ** \\
      RBI & 47.46 & 37.06 & -10.4 & -21.91 & 0.0635 & * \\
      SB & 5.16 & 3.68 & -1.48 & -28.68 & 0.2769 &  \\
      TB & 151.74 & 116.6 & -35.14 & -23.16 & 0.0449 & ** \\
    \end{tabular}
  \end{center}
\end{table}

These results on their own do a decent job of telling a story. All eight of these statistical categories saw a precipitous drop 
from 2021 to 2022, with all of them falling over 15% year-over-year. Four of the eight categories contain a rejection of the 
null hypothesis at alpha = 0.075. Two of the eight, runs and home runs, reject the null hypothesis at alpha = 0.025. However, 
baseball pundits tend to scoff at analyses that only include these basic counting statistics. There’s vastly much more to the 
offensive game, they claim, than just hits and runs.

The pundits aren’t entirely wrong. A better measure of player performance might likely be the value-added 
statistics that we mentioned earlier. These statistics take a holistic approach to the game, attempting to summarize the true 
impact that a player might have on the games they participate in. The next table will contain the results of the analysis of 
variance for our two value-added statistics. The same column headings used in the previous table are used here, with the same 
functionality. Bear in mind that these statistics are dependent on game participation- they are more likely to increase the 
more chances a player has- making them still technically counting statistics.

\begin{table}[h!]
    \begin{center}
      \caption{Value-Added Statistics}
      \label{tab:table2}
      \begin{tabular}
        \textbf{Stat} & \textbf{2021} & \textbf{2022} & \textbf{Diff.} & \textbf{PctgChg} & \textbf{P-Value} & \textbf{Signif.}
        \hline
        RAR & 13.96 & 8.02 & -5.94 & -42.55 & 0.1043 &  \\
        WAR & 1.332 & 0.754 & -0.578 & -43.39 & 0.1191 & \\
      \end{tabular}
    \end{center}
\end{table}

These analyses fall victim to the lofty standards of statistical significance. However, the same large drops can be seen in these 
value-added statistics. Both categories fell over 40% from 2021 to 2022, even if neither was significant at the level alpha = 0.10. 
It should be noted, however, that both P-values came extremely close to being significant at that level. While our ANOVA may not 
have produced any definitive conclusions here, we can still see that our players, on average, added nearly six more runs to their 
teams in 2021 than in 2022. 

You may be asking: what’s going on with these huge year-over-year changes? After all, the fifty-player sample for 2022 is just 
the fifty-player sample for 2021 with a mere 365 days of added age. This is where it feels appropriate to mention the theories 
of John Cassidy and Kenneth Lehn that were mentioned in the introductory section. Are players that switch teams simply less fit 
to play? Our health statistics are able to gauge if there are significant differences in game participation from one year to 
the next. The next table will contain the results of the analysis of variance for our two health statistics. The same column 
headings used in the previous table are used here, with the same functionality. 

\begin{table}[h!]
    \begin{center}
      \caption{Health Statistics}
      \label{tab:table3}
      \begin{tabular}
        \textbf{Stat} & \textbf{2021} & \textbf{2022} & \textbf{Diff.} & \textbf{PctgChg} & \textbf{P-Value} & \textbf{Signif.}
        \hline
        GM & 110.3 & 86.42 & -23.88 & -21.65 & 0.0069 & *** \\
        IL & 27.24 & 30.32 & 3.08 & 11.31 & 0.6698 &  \\
      \end{tabular}
    \end{center}
\end{table}

After looking at this table, it seems we may have found our culprit. The difference in game participation between 2021 and 
2022 for our players, as well as its accompanying P-value, is staggering. The null hypothesis is rejected for games played at 
alpha = 0.01. The drop from 110 games to 86 games is substantial, especially in a season that is only 162 games long. The high 
P-value for days spent on the injured list is disappointing from a null hypothesis perspective, but we can still learn from 
this finding. This simply shows that we can’t definitively say that major catastrophic injuries are the reason why player 
participation dropped so heavily from 2021 to 2022.

While we surely can pull information from these statistical findings, we wouldn’t be doing our due diligence if we didn’t run 
tests that control for game participation. We’ve established that players who switched teams played vastly less on their new 
teams, but how did they perform when they actually played? To answer this question, we have to use our proportion statistics. 
As stated before, these statistics do not necessarily increase with increasing opportunity. Rather, they are moving indicators 
of how any player is faring in their time on the field, however substantial that time may be.

It is important to note a few things before we change course. These proportion statistics are more prone than counting 
statistics to anomalies due to insufficient sample size. If you check the leaderboards for these categories just a few weeks 
into the season, you will see players you’ve never heard of sitting squarely in the top twenty. However, the standard error of 
these statistical categories is greatly decreased given a large sample size. The central limit theorem states that as the 
season goes on and sample size increases, the numbers will slowly approach what they are bound to be by the end of the 
season. I believe that the hazard of inadequate sample size is prevented by employing the pooled-proportion strategy that was 
mentioned earlier. With the exception of stolen base percentage, all of these proportions have denominators of over 14,000, 
which should be more than enough to avoid the pitfalls of potential wayward results and statistical anomalies.

The next table will contain the results of the analysis of variance for our proportion statistics. The column headings here 
are slightly different from before. The first column describes the statistic of interest. The second column contains the 
numerator, denominator, and final calculated proportion for that statistic in 2021. The third column contains the numerator, 
denominator, and final calculated proportion for that statistic in 2022. The fourth column contains the P-value for the difference 
in the two proportions. The final column contains the significance marking for the statistic. A statistic with one asterisk 
indicates rejection of the null hypothesis at alpha = 0.10. A statistic with two asterisks indicates rejection of the null 
hypothesis at alpha = 0.05. A statistic with three asterisks indicates rejection of the null hypothesis at alpha = 0.01.

\begin{table}[h!]
    \begin{center}
      \caption{Proportion Statistics}
      \label{tab:table4}
      \begin{tabular}
        \textbf{Stat} & \textbf{2021} & \textbf{2022} & \textbf{P-Value} & \textbf{Signif.}
        \hline
        AVG. & 4391/17591 = .24962 & 3577/14557 = .24572 & 0.4237 &  \\
        SLG. & 7587/17591 = .43130 & 5830/14557 = .40049 & <0.0001 & *** \\
        HR PCT & 742/19853 = 3.737 & 491/16198 = 3.031 & 0.0002 & *** \\
        BB PCT & 1831/19853 = 9.223 & 1310/16198 = 8.087 & 0.0001 & *** \\
        SO PCT & 4413/19853 = 22.228 & 3568/16198 = 22.027 & 0.6455 &  \\
        SB PCT & 258/332 = 77.711 & 184/244 = 75.410 & 0.5157 &  \\
      \end{tabular}
    \end{center}
\end{table}

We can see here that, even when adjusting for the fact that our players were playing less in 2022, their performance took a 
tumble after moving to their new team. There are remarkably low P-values for the differences in slugging percentage, home run 
percentage, and walk percentage. The difference in the three other statistical categories was not significant at any alpha 
level below 0.10, but a drop-off in performance can still be seen. The batting average for our players dropped nearly four 
points year-over-year (the batting average for the entire league only saw a dropoff of less than one point) and our players 
found slightly less success when trying to steal bases as well.

\section{Discussion}

Our first table showed statistically significant declines in runs, home runs, runs batted in, and total bases. These categories 
share a sort of commonality in that they are all statistics that rely on a hitter being able to have lots of swing power. Home 
runs are almost exclusively considered a sign of power, and our players hit, on average, five fewer home runs in 2022 than they 
did in 2021. There could be several possible explanations for this finding. A baseball team plays half of its games at its home 
field; when a player switches teams, they are invariably switching the venue in which they play a large chunk of their games. 
Ballpark dimensions vary from stadium to stadium, with the fences differing in height, distance from home plate, and angle. A 
good example of this comes from Trevor Story, one of our 50 sampled players and the one from the anecdote at the beginning of 
my introduction. Coors Field, his old field in Denver, measures 390 feet from home plate to the center field wall. Fenway Park, 
his new field in Boston, measures 420 feet to center field- a whopping 30-foot difference. Hitting a baseball 420 feet is a 
tall task. Players who have played in Boston for years will tell you they never try to hit a ball to center field, lest they 
waste their power trying to get the ball over one of the farthest walls in the entire league. Conversely, Story, playing his 
first season in Boston, is likely unfamiliar with the quirks and oddities that come with being a batter in a particular ballpark 
with unique dimensions. Our sample of 50 players all had to adapt to hitting in a new ballpark, forcing them to restructure 
their entire approach at the plate. This seismic disruption in their play style would also explain why these players were 
slightly less valuable in 2022 than in 2021, as shown in our second table. 

I would be remiss if I didn’t also mention that in general, moving teams is as massive a change off the field as it is on the 
field. Anybody who has transplanted themselves into a new city will agree that a massive adjustment period is necessary  
to get one’s feet underneath them in a new environment. Shortstop Andrelton Simmons, one of our sample subjects, made 
the transition from small-market Minneapolis to the Windy City of Chicago, the third-largest media market in the United States. 
The aforementioned Story went from Denver, a city over 5,000 feet above sea level, to Boston, which is only a hair over 100 
feet above sea level. It is not difficult to fathom a potential link between player performance issues and the enormous 
life-altering decisions that directly precede them. 

Of course, it must be discussed that a sizable portion of the drop-off can be explained by our players’ lack of physical 
fitness in 2022. As could be seen in our third table, game participation fell by nearly 25 games per player from 2021 to 2022. 
This difference was deemed empirically significant, and there is a theoretical explanation for this finding as well. Chapter 
12 of John Cassidy’s book How Markets Fail is entitled “George Akerlof’s Market for Lemons”, wherein he explains that unequal 
access to information plagues almost every sector of our economy. This includes the market for baseball free agents, which 
Cassidy explains suffers from the problem of hidden injury information. He posits that “a player’s original team knows more 
about his health than a potential acquirer, and the decision to release him to free agency may be a sign that it has concerns 
about his health” (Cassidy 154). This hypothesis is consistent with the previously mentioned studies of Garcia et al. which 
found that teams are more willing than ever to part with their own players. While the drop in games played doesn’t necessarily 
reveal a regression in talent or skill, a player needs to be healthy in order to be useful. Our value statistics are tied to 
game participation because a player that is riding the bench with back problems isn’t valuable at all. Teams are, generally, 
required to pay players their massive salaries regardless of how healthy they are, making this trend of bad health in newly 
signed players troubling for big spenders. Teams see big free agency contracts as an investment, and a player can’t provide a 
return on that investment unless they’re truly fit to play. It should also be mentioned the effect that switching organizations 
might have on a player’s health. Even if there aren’t any nagging hidden issues like the ones Cassidy hypothesized about, a 
new team means a new training staff and a new set of team doctors. Teams aren’t owed any extensive medical records when they 
sign a player, so medical staffs have very little to work from when treating a newly acquired player.

Arguably the most troubling finding for our players, even more so than their dismal health, was their inability to meet 
expectations when they were on the field. Our fourth and final table contained statistical categories whose drop-offs could 
not be explained by not playing as many games in 2022. Every category decreased in some way except for strikeout percentage, 
with slugging percentage, home run percentage, and walk percentage decreasing significantly. It seems as though between power, 
contact, speed, and discipline, power is the performance tool that appears to have evaporated the most in our players. The drop 
in slugging percentage and home run percentage matches up with the drop in other power-related categories from table one. 
Hitting power is a complicated concept. Gratuitous strength doesn’t always correlate with good power, but it certainly does 
play a part. Perfect swing form and unimpeachable timing ability are inarguably more important when it comes to launching the 
baseball- strength means nothing if you cannot get the barrel of the bat on the ball. A baseball swing can be thought of as a 
symphony of moving body parts. If a batter develops even the slightest hitch in their swing, the entire orchestra is thrown out 
of balance and the player stands no chance of hitting the ball squarely. These hitches usually develop from, coincidentally, 
nagging injuries to the back, neck, and extremities. With so many of the players in our sample battling long-term ailments in 
2022, it is not difficult to see how their power was essentially zapped when compared to the season before. These weeks-long 
bouts with injury can cause major performance slumps, draining a player’s confidence and setting them up for failure. Baseball 
is partially a mental game- step up to the plate with a pessimistic mindset, and you’re almost guaranteed to fail. Early-season 
struggles, especially with a brand-new team, can easily spiral into season-long, emotionally torturous campaigns that leave 
fans wondering what exactly the team signed up for when they shelled out millions of dollars.

All of this cynicism is not to say that our players can’t turn their fortunes around. The numbers don’t lie: the players in 
our sample saw a downturn in essentially every single statistical category we decided to investigate. However, we have seen 
players take their lumps with a new team and persevere. Kyle Schwarber, one of the sampled players, saw one of the largest 
year-over-year decreases in batting average from 2021 to 2022. But when his Philadelphia Phillies made their improbable run 
to the World Series this November, Schwarber reversed course and smashed 6 home runs in 17 games, giving Phillies fans hope for 
a brighter campaign ahead. Baseball is, and will continue to be, an endearingly unpredictable sport. Players that switch teams 
will likely always have their growing pains, but the fans that stick it out might just be lucky enough to bear the fruit when 
those players bloom.

\bibliographystyle{chicago}
\bibliography{citations.bib}

\section{Acknowledgments}

Thank you to the University of Connecticut Department of Statistics faculty, and specifically to Dr. Jun Yan, for their 
contribution to and assistance with the writing of this paper.

\section{About The Student Author}

Jason Boghosian is an expected graduate of the University of Connecticut with a B.A. in Statistics and a minor in both 
Public Policy and Urban Studies. He will begin his postgraduate studies in August of 2023 when he pursues a Master of Arts 
in Survey Research at the University of Connecticut.

\section{Press Summary}

The purpose of this study was to examine the effects that switching teams had on professional baseball players in 2022. The 
hope was to investigate whether there were statistical grounds for the claim that players who change organizations generally 
perform worse on their new team. This study focused on the difference from 2021 to 2022 in 18 statistical categories for 50 
sampled MLB players that switched teams between the two seasons. After running statistical tests, the findings implied  
that players’ fitness level, health, and overall hitting performance appeared to drastically regress after switching teams.
